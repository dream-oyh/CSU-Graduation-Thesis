%!TEX root = ../csuthesis_main.tex
\begin{acknowledgements} 

大学四年是我作为“人”,真正树立起自己价值观、人生观、世界观的地方,大一的时候我对自己说:“要活的精彩,活的灿烂,高级而鲜活。”站在当下,我可以说我做到了。家乡的梅花快要开了,影影绰绰的枝丫里藏着亭台和楼阁,风里带着清新的泥土气息;夺目的烈日下,飞檐映照着粼粼水波,北京的园林里我和朋友聊着不确定的未来与人生;池塘边闪过一只鹤的影子,清冷的月光下她们埋葬了诗与花的魂灵;窗外,雪下的正大,我们坐在屋内,手中摇晃的茶杯里是三炮台,嘴里是刚嚼完水焯羊肉的醇香……我们对春夏秋冬的体验串联起了生活的全部,也构成了我对这个世界的感知,如此,便可谓不负韶华了。

没有一个人能只靠自己就走到当下。首先要感谢的是阳劲松老师,他为我提供了这次和浙大一起做毕设的机会,也在这个过程中为我的毕业论文做出指导,从大一的工图课上就认识了阳老师,认同于他的风趣、友善和体贴学生,最后也由他带队完成了我的生产实习,再次感谢阳老师。

其次,感谢中南大学四年来对我的培养,感谢珊姐和健哥的温柔以待,感谢设三支部所有同志的互相支持与帮助,感谢新宣里一起共事的小伙伴,很幸运能够遇见这么优秀的你们,与你们一起成长,是我的荣幸。

感谢浙大的李德骏教授、林鸣威老师和楚曙光师兄,他们以前瞻性的学术视角为我拟定了本次毕设的选题,在我遇到困难时候的安慰和帮助都给了我继续攻克难题的信心和勇气,同时也提供了我充足的学术指导,没有他们我无法独自完成这个课题。

感谢彭勇老师和向国梁师兄在本科期间对我的指导,通过交科赛我真正获得了科研的经验,也借此机会发表了第一篇SCI论文,他们是我学术研究的领路人,这段经验也成为我之后保研到浙大一块重要的敲门砖。

感谢我最好的三位朋友,薛奕超、刘宣乐、李嘉贤,薛总以幽默风趣的生活态度和具有洞察力的社会理解带给我力量和成长,刘以极其丰富的计算机技能储备带领我跨入编程和算法的门扉,嘉贤则是大学里一起作战的伙伴,每个竞赛和成长的路上,我们都互相陪伴。

感谢父母把我培养成人,不仅从物质也从精神层面支持着我走到了这里,家庭永远是人生路上最坚实的依靠。

感谢本科努力奋斗的自己,“佼佼者从来来去自由”。

敬过去的四年,敬即将开启的未来。

\end{acknowledgements}
