%!TEX root = ../csuthesis_main.tex
% 设置中文摘要
\keywordscn{参数辨识\quad 扩展卡尔曼滤波\quad MPC控制算法}
%\categorycn{TP391}
\begin{abstractzh}

随着“海洋强国”的发展,海底运输、海底勘探越来越成为全球性的热门研究问题,近年来,水下机器人由于其灵活性和可控的自主性而被广泛应用于水下运输、海底探测等专业领域,本文针对用于水下运输的遥控运载机器人进行建模和仿真研究。

本文的研究对象为BlueROV2这款开源机器人,根据BlueROV2的相关结构特性,定义运动参数,建立载体坐标系和惯性坐标系,获得运动学方程,又根据水下环境和BlueROV2电机分布对水下机器人进行受力分析,得到动力学方程。随后,本文将BlueROV2的六自由度运动解耦为水平面运动模型和竖直面运动模型,分别对两个运动模型设计最小二乘算法,辨识出水动力系数;此外,为解决各个自由度之间的强耦合关系,设计基于扩展卡尔曼滤波器(EKF)的辨识算法,搭建包含控制系统、测量系统的仿真结构,对水动力系数进行辨识。最后,本文通过UUV Simulator搭建水下环境和MPC控制框架,设计圆形和双纽线形轨迹跟踪任务,分别验证最小二乘算法和EKF算法辨识出的参数在轨迹跟踪任务下的表现,得出结论,圆形轨迹下,EKF算法较最小二乘算法表现更好,而双纽线轨迹下,两种算法的轨迹追踪任务表现相当。


\end{abstractzh}