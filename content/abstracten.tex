%!TEX root = ../csuthesis_main.tex
\keywordsen{System Identification\ \ Extended Kalman Filter\ \ MPC Controller}
\begin{abstracten}

With the development of a "maritime power", underwater transportation and seabed exploration have increasingly become hot research topics. In recent years, underwater robots have been widely applied in professional fields due to their flexibility and controllable autonomy. This paper conducts modeling and simulation research on the Remotely Operated Vehicle (ROV) used for underwater transportation. 

The subject of this paper is the open-source robot BlueROV2. Based on the relevant structural characteristics of BlueROV2, we define the motion parameters, establish the inertial frame and body-fixed frame, and obtain the kinematic equations. Then, we analyze the force analysis of the underwater robot with the underwater environment characteristics and the motor distribution of BlueROV2, and finally, obtain the dynamic equations. Subsequently, the six-dofs motion of BlueROV2 is decoupled into the horizontal motion model and the vertical motion model. The least squares algorithm is designed for the two motion models to identify the hydrodynamic coefficients. In addition, to solve the strong coupling relationship among the degrees of freedom, an identification algorithm based on the extended Kalman filter (EKF) is designed, and a simulation structure including the control system and the measurement system is built to identify the hydrodynamic coefficients. Finally, the underwater environment and the MPC control framework are built in UUV Simulator, which is a simulation tool based on ROS and Gazebo. We design the circular and lemniscate trajectory tracking tasks to verify the different performance between the two parameter groups, identified by the LS and EKF, respectively. The conclusion is drawn that the EKF algorithm performs better than the least squares algorithm in the circular trajectory, while the performance of the two algorithms in the lemniscate trajectory tracking task is comparable.

\end{abstracten}