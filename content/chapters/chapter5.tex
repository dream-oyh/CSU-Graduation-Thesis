%!TEX root = ../../csuthesis_main.tex
\chapter{结论}
\section{论文工作总结}

随着全球对海洋资源的勘探愈发重视,水下航行器迅速发展,被广泛应用在深海勘探、海洋开发等领域,有效致力于“海洋强国”建设,而运动控制精度和稳定性是水下机器人能够正常工作的基础。本文通过刚体动力学理论,对用于水下运输的遥控运载机器人运动控制进行研究,主要内容如下:

\begin{enumerate}
    \item 本文简要概述了课题的研究背景与意义,对国内外在ROV建模仿真技术、仿真平台技术与运动控制技术三方面的研究现状进行了简要分析。
    \item 本文基于刚体动力学理论,介绍了BlueROV2这款开源水下机器人的结构设计,定义载体坐标系和惯性坐标系,定义运动参数,获得了载体坐标系和惯性坐标系下速度的转换方程,分析了BlueROV2的电机分布与受力情况,完成了对BlueROV2进行运动学模型和动力学模型的建立。
    \item 本文完成了ROV辨识算法的设计。由于方程个数与待辨识参数个数不兼容,故将BlueROV2六自由度的运动解耦为水平面运动和竖直面运动,设计最小二乘算法分别对两个运动模型中的参数进行辨识;除此之外,设计扩展卡尔曼滤波器直接对BlueROV2六自由度运动过程进行辨识,搭建了包含控制系统和测量系统的仿真结构,通过计算增广状态向量最终的收敛值辨识动力学模型参数。
    \item 本文完成了对ROV控制框架的设计,基于UUV Simulator搭建了BlueROV2的仿真环境与基于MPC的控制框架,设计了圆形参考轨迹和双纽线参考轨迹,分别对比了由最小二乘算法辨识和由EKF算法辨识出的参数在不同轨迹跟踪任务下的表现,得到结论在圆形轨迹下,EKF算法的表现要比最小二乘算法好,而在更为复杂的双纽线轨迹下,最小二乘算法和EKF算法辨识结果相当。
\end{enumerate}

\section{展望}

本文针对用于水下运输的遥控运载机器人建模与仿真进行研究,但所研究与设计算法依然存在不足,无法被部署在真实的工程应用中,本文的研究内容展望如下:

\begin{enumerate}
    \item 本文的最小二乘算法和EKF卡尔曼滤波算法虽然都辨识出了模型参数,但是其只能满足全局最优,却无法做到使具体每一个水动力参数都收敛至真实值,这对后续的轨迹跟踪和控制带来了很大的难题。因此,后续的研究会考虑结合深度学习与强化学习等先进计算方法,提出基于数据驱动的参数辨识新思路。
    \item 本文主要针对BlueROV2主体的运动控制进行研究,未曾考虑由于“取样”“加装机械臂”等带来的动力学参数改变,后续的研究应该针对于复杂系统进行辨识,获得水下机器人各个模块之间的耦合关系。
    \item 本文的研究目前仅停留于仿真阶段,并没有将具体算法部署至实际的水下机器人上,后续的研究应该考虑结合嵌入式等物联网技术,将算法部署到实际的机器人控制终端上开展实验测试,以应对更为真实、多变和复杂的水下环境。
\end{enumerate}